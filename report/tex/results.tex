\documentclass[../report.tex]{subfiles}

\begin{document}
\chapter{Results}

\section{Comparision Between SSC Algorithms}
EWKM, LEKM, and, FSC were compared on their best hypterparameters according to the \begin{color}{red}cost function\end{color} on different values of \textit{k} (50, 100, 500). Figure \ref{ewkm-weights} shows the weight grid of EWKM on different values of \textit{k}. Figure \ref{lekm-weights} shows the weight grid of EWKM on different values of \textit{k}. Figure \ref{lekm-weights} shows the weight grid of EWKM on different values of \textit{k}.

Table ... shows the different purity of the different algorithms on different \textit{k}

\subsection{EWKM}
EWKM was first ran on between values of $\gamma = [0.5-3]$, the recommended range as described in \cite{Jing2007, wskm2014hz}. The negative entropy ended up being larger than the cost function resulting in immediate convergence.

On the next iteration of tests, $\gamma$ was lowered until immediate convergence did not occur, the best $\gamma$ parameter and purity is shown in . The resulting Purity is lower than other algorithms for all $k$. All tested $\gamma$'s resulted in a weight distribution, in which all clusters only had one significant dimension.

Table ... shows diff

\subsection{LEKM}
LEKM was ran on $\gamma$'s ranging from 0.5 to 3. The LEKM did not have the same problem as EWKM on the given range. The best performing gamma is shown in ... . The scores were comparable to FSC. All $\gamma$'s resulted in normal distribution of feature weights. Higher values of $\gamma$ resulted in a smaller standard deviation, leading to a more uniform distribution.


Table ... shows diff

\subsection{FSC}
$\beta$ was recommended to be set to 2.1 and $\epsilon$ to 0.0001 \cite{Gan2006}. In our tests, we tested $\beta$'s between 1.5 and 30 with $\epsilon$ set to 0.0001. Based on purity $\lim_{\beta}{\inf}$ led to the best score. The score of FSC was better than EWKM and similar to LEKM. $\beta = 1.5$ led to a single dominant feature, similar to EWKM. For 

Table ... shows diff

\begin{figure}
     \centering
     \begin{subfigure}[b]{0.3\textwidth}
         \centering
         \includegraphics[width=\textwidth, height=40mm, keepaspectratio]{../../repos/lekm/clusters/z/K_500_L_0-5/plots/heatmap.png}
         \caption{$k=500 \text{, } \gamma=2$}
         \label{fig:y equals x}
     \end{subfigure}
     \hfill
     \begin{subfigure}[b]{0.3\textwidth}
         \centering
         \includegraphics[width=\textwidth, height=40mm, keepaspectratio]{../../repos/lekm/clusters/z-100/K_100_L_2-0/plots/heatmap.png}

         \caption{$k=100 \gamma=2$}
         \label{fig:three sin x}
     \end{subfigure}
     \hfill
     \begin{subfigure}[b]{0.3\textwidth}
         \centering
         \includegraphics[width=\textwidth, height=40mm, keepaspectratio]{../../repos/lekm/clusters/z-50/K_50_L_0-5/plots/heatmap.png}
         \caption{$k=50 \gamma=0.5$}
         \label{fig:five over x}
     \end{subfigure}
        \caption{Three simple graphs}
        \label{fig:three graphs}
\end{figure}

\begin{figure}
     \centering
     \begin{subfigure}[b]{0.3\textwidth}
         \centering
         \includegraphics[width=\textwidth, height=40mm, keepaspectratio]{../../repos/lekm/clusters/z/K_500_L_2-0/plots/weights-v2.png}
         \caption{$k=500 \text{, } \gamma=2$}
         \label{fig:y equals x}
     \end{subfigure}
     \hfill
     \begin{subfigure}[b]{0.3\textwidth}
         \centering
         \includegraphics[width=\textwidth, height=40mm, keepaspectratio]{../../repos/lekm/clusters/z-100/K_100_L_2-0/plots/weights-v2.png}

         \caption{$k=100 \gamma=2$}
         \label{fig:three sin x}
     \end{subfigure}
     \hfill
     \begin{subfigure}[b]{0.3\textwidth}
         \centering
         \includegraphics[width=\textwidth, height=40mm, keepaspectratio]{../../repos/lekm/clusters/z-50/K_50_L_0-5/plots/weights-v2.png}
         \caption{$k=50 \gamma=0.5$}
         \label{fig:five over x}
     \end{subfigure}
        \caption{Three simple graphs}
        \label{fig:three graphs}
\end{figure}

\begin{table}[h]
\begin{center}
\begin{tabular}{lr@{\hspace{0.2in}}rrrrrrrr}
  \hline\noalign{\smallskip}
  && \multicolumn{2}{c}{\tbtitle{EWKM}} && \multicolumn{2}{c}{\tbtitle{LEKM}} && \multicolumn{2}{c}{\tbtitle{FSC}} \\
  \noalign{\smallskip} \cline{3-4} \cline{6-7} \cline{9-10}
        \multicolumn{1}{c}{\textit{k}} && \tbtitle{$\gamma$} & \tbtitle{Purity} && \tbtitle{$\gamma$} & \tbtitle{Purity} && \tbtitle{$\beta$} & \tbtitle{Purity}\\
        \hline
        \multicolumn{1}{r|}{50 } && 0.005 & 28.2\% && 0.0 & 0.0 && 2.1* & 0.0\\
        \multicolumn{1}{r|}{100} && 0.005 & 28.9\% && 2.1 & 40.6\% && 2.1* & 0.0\\
        \multicolumn{1}{r|}{500} && 0.001 & 28.6\% && 2.2 & 45.2\% && 2.1* & 43.8\\
\end{tabular}
\end{center}
\caption{Best hyperparameter ($\gamma$, $\beta$) given \textit{k}}
\end{table}


\section{K-means}

\section{External Evaluation}
8 contestents were


\end{document}

