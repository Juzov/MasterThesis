\documentclass[../report.tex]{subfiles}

\begin{document}
\chapter{Results}
The results of this chapter have been divided into two sections: \textit{General Results} and \textit{Evaluation Results}. General Results shows how different algorithms performed in terms of purity, convergence speed and feature weight distribution. The section ends with a summary showing the best parameters and purity scores for each algorithm. Evaluation Results presents the expert evaluation scores of k-means, the best performing SSC-algorithm, and playlists, through \cref{table:rating}.

\section{General Results}
\label{section:general}
Below sections describe the results of k-means, EWKM, LEKM, and FSC with different parameters on the dataset. The section is summarized by \cref{table:purity}, which shows the purity of the best performing parameters for each algorithm.

To avoid repetitiveness algorithm-specific figures are shown only for $k=500$. The choice of $k=500$ is justified as it is the best performing $k$ for algorithms in regards to purity (see \cref{res:summary}).

\subsection{EWKM}
\label{subsection:ewkm}

\Cref{fig:ewkm-purity} presents the purity given different values of $\gamma$ for EWKM. From the figure we see a trend of the purity slowly decreasing until a value of $\gamma = 2$, where a steep increase occurs to a score of $0.44$.

\Cref{fig:ewkm-iterations} shows the corresponding amount of iterations until convergence. The amount of iterations needed to converge is decreasing when the value of $\gamma$ is increasing. Immediate convergence occurs for values of $0.05$ and higher.

\Cref{fig:ewkm-restarts} shows the amount of restarts needed to generate non-empty cluster partitions. Most value selections needed zero or one restarts. The outlier was the choice of $0.01$, which required $10$ restarts. Selections larger than $2.0$ resulted in zero restarts.

Immediate convergence (with no restarts) --- as shown in the figures to be $2.0$ and higher, were caused by the Shannon entropy being more negative than the total dispersion within clusters, making the objective function negative.

\Cref{fig:ewkm-meshgrid} represents the feature weights of each cluster as colored grids (often referred to as a meshgrid) for various choices of $\gamma$. The grids in \cref{fig:ewkm-meshgrid} show the weight of a specific feature for a cluster as a colored rectangle. Yellow denotes a high feature weight, while purple denotes a low value. A cluster's feature weight vector is a horizontal line in the grid. For all the tested values of $\gamma$ that did not result in immediate convergence, the weight distribution of the clusters were similar. One feature was often dominant, i.e. had a high weight (around 1) while the rest had feature weights near zero. The grids do however, show a trend of less dominant features being produced with larger values of $\gamma$.

% The results of dominant features are unlike what is presented in \cite{Jing2007}
\newpage
\begin{figure}[H]
  \centering
  \begin{subfigure}{0.7\textwidth}
    \begin{center}
      \includegraphics[width=\linewidth, keepaspectratio]{../../repos/ewkm/clusters/500-plots-mannen-riktiga-svar-pa-allt-9/plots/gamma-purities.png}
      \caption{Purity accuracy of EWKM \\ given various values of $\gamma$ ($k=500$).}
      \label{fig:ewkm-purity}
    \end{center}
  \end{subfigure}
  \medskip
  \centering
  \begin{subfigure}{0.7\textwidth}
    \begin{center}
    \includegraphics[width=\linewidth, keepaspectratio]{../../repos/ewkm/clusters/500-plots-mannen-riktiga-svar-pa-allt-9/plots/gamma-iterations.png}
    \caption{Iterations until convergence of EWKM given various values of $\gamma$ ($k=500$).}
    \label{fig:ewkm-iterations}
    \end{center}
  \end{subfigure}
  \medskip
  \centering
  \begin{subfigure}{0.7\textwidth}
    \begin{center}
    \includegraphics[width=\linewidth, keepaspectratio]{../../repos/ewkm/clusters/500-plots-mannen-riktiga-svar-pa-allt-9/plots/gamma-restarts.png}
    \caption{Restarts of EWKM given various values of $\gamma$ ($k=500$).}
    \label{fig:ewkm-restarts}
    \end{center}
  \end{subfigure}
\end{figure}
\newpage


% The results of the figures, shows one dominant feature weight (with a value around 1.0) in each cluster for .

% , is different from the expected results of a normalized global feature space as described in \cite{Jing2007} for the algorithm.

\newpage
\begin{figure}[H]
  \centering
  \begin{subfigure}{0.75\textwidth}
    \begin{center}
      \includegraphics[trim={30pt 20pt 30pt 30pt},clip, width=\linewidth, keepaspectratio]{../../repos/ewkm/clusters/500-plots-mannen-riktiga-svar-pa-allt-9/K_500_L_0-0005/plots/heatmap.png}
      \caption{$\gamma=0.0005$}
    \end{center}
  \end{subfigure}
  % \vspace*{-10pt}
  \centering
  \begin{subfigure}{0.75\textwidth}
    \begin{center}
      \includegraphics[trim={30pt 20pt 30pt 30pt},clip, width=\linewidth, keepaspectratio]{../../repos/ewkm/clusters/500-plots-mannen-riktiga-svar-pa-allt-9/K_500_L_0-05/plots/heatmap.png}
      \caption{$\gamma=0.05$}
    \end{center}
  \end{subfigure}
  \vspace*{-2pt}
  \centering
  \begin{subfigure}{0.75\textwidth}
    \begin{center}
      \includegraphics[trim={30pt 20pt 30pt 30pt},clip, width=\linewidth, keepaspectratio]
      {../../repos/ewkm/clusters/500-plots-mannen-riktiga-svar-pa-allt-9/K_500_L_0-5/plots/heatmap.png}
      \caption{$\gamma=0.5$}
    \end{center}
  \end{subfigure}
  \caption{Feature-weight meshgrid of EWKM ($k=500$). A row represents a cluster subspace (feature-weight vector), and a column represents a feature-weight along clusters.}
  \label{fig:ewkm-meshgrid}
\end{figure}
\newpage


% Figure \cref{ewkm-weights} shows the weight grid of EWKM on different values of \textit{k}. Figure \cref{lekm-weights} shows the weight grid of EWKM on different values of \textit{k}. Figure \cref{lekm-weights} shows the weight grid of EWKM on different values of \textit{k}. The convergence delta was set to $0.005$ i.e. $\frac{ \text{prev_disp} - \text{cur_disp} }{ \text{prev_disp} } < 0.005$

% On the next iteration of tests, $\gamma$ was lowered until immediate convergence did not occur, the best $\gamma$ parameter and purity is shown in . The resulting Purity is lower than other algorithms for all $k$. All tested $\gamma$'s resulted in a weight distribution, in which all clusters only had one significant dimension.

% Table ... shows diff

% Figure \cref{ewkm-weights} shows the weight grid of EWKM on different values of \textit{k}. Figure \cref{lekm-weights} shows the weight grid of EWKM on different values of \textit{k}. Figure \cref{lekm-weights} shows the weight grid of EWKM on different values of \textit{k}. The convergence delta was set to $0.005$ i.e $\frac{ prev_disp - cur_disp }{prev_disp } < 0.005$

% Table ... shows the different purity of the different algorithms on different \textit{k}

\subsection{LEKM}
The purity scores of LEKM are shown in \cref{fig:lekm-purity}. With higher values of gamma the purity is increasing until a peak purity is reached at $\gamma = 1.4$ with a score of $45.5\%$. For values $\geq 2.6$ the purity decreases steeply to values of $43.8\%$, these values correspond to values of immediate convergence.

The amount of iterations until convergence is shown in \cref{fig:lekm-iterations}. Iterations kept increasing with $\gamma$ until a value of $2.6$. The largest amount of iterations needed was found at $\gamma=2.2$ with 20 iterations. Values of $2.6$ and larger resulted in an immediate convergence. Compared to EWKM, the $\gamma$'s of LEKM are larger when immediate convergence occurs.

% \textit{LEKM} performed significantly better than EWKM in terms of purity for most values of $\gamma$. Similar to EWKM, $\gamma$ of a certain size resulted in immediate convergence, due to the same reason; A negative entropy more negative than the within cluster variance.
% The interval of $\gamma$'s ranging from 0.5 to 10 were tested on LEKM. Generally,

% \Cref{fig:ewkm-meshgrid} represents the feature weights of each cluster as colored grids (often referred to as a meshgrid) for various choices of $\gamma$. The grids in \cref{fig:ewkm-meshgrid} show the weight of a specific feature for a cluster as a colored rectangle. Yellow denotes a high feature weight, while purple denotes a low value. A cluster's feature weight vector is a horizontal line in the grid. For all the tested values of $\gamma$ that did not result in immediate convergence, the weight distribution of the clusters were similar. One feature was often dominant i.e. had a high weight (around 1) while the rest had feature weights near zero. The grids do however, show a trend of less dominant features being produced with larger values of $\gamma$. The results of dominant features are unlike what is presented in \cite{Jing2007}

The feature weight grids of LEKM is shown in \cref{fig:lekm-meshgrid}, note that colors represent different values for each grid. We see that there are multiple features representing the cluster through the figures. Certain features are dominant in very few clusters (visible as a purple-dominant vertical line in the grid) while others are dominant in most clusters (visible as a yellow-dominant vertical line in the grid). The highest weights are also small compared to EWKM, but are still, larger than the least important features of the cluster. The dispersion between small and large weight values decrease with a higher $\gamma$.

Another perspective of how the weights differ given various $\gamma$ is shown in \cref{fig:lekm-distribution}. The histograms of the figure are the distribution of values of all feature weights in all clusters. The feature distribution of LEKM resembles a slightly leaning hill with an J-shaped peak, similar to what is mentioned in \cite{Jing2007}. The deviation of weight values decrease with an increase of the $\gamma$ value, leading to a more and more uniform distribution.

\begin{frame}

\begin{figure}
\makebox[\linewidth][c]{%
\begin{minipage}{.6\textwidth}
  \includegraphics[width=\linewidth, keepaspectratio]{../../repos/lekm/clusters/500-plots-mannen-riktiga-svar-pa-allt/plots/gamma-purities.png}
  \caption{Purity accuracy of LEKM \newline given various values of $\gamma$ ($k=500$).}
  \label{fig:lekm-purity}
\end{minipage}\hfill
\begin{minipage}{.61\textwidth}
  \includegraphics[width=\linewidth, keepaspectratio]{../../repos/lekm/clusters/500-plots-mannen-riktiga-svar-pa-allt/plots/gamma-iterations.png}
  \caption{Iterations until convergence \newline of LEKM given various values of $\gamma$ ($k=500$).}
  \label{fig:lekm-iterations}
\end{minipage}
}
\end{figure}

\end{frame}

\newpage
\begin{figure}[H]
\makebox[\linewidth][c]{%
\begin{subfigure}[b]{.7\textwidth}
\centering
\includegraphics[trim={30pt 20pt 30pt 30pt},clip, width=\linewidth, keepaspectratio]
{../../repos/lekm/clusters/500-plots-mannen-riktiga-svar-pa-allt-new-distance/K_500_L_0-5/plots/heatmap.png}
\caption{$\gamma=0.5$}
\end{subfigure}%
\begin{subfigure}[b]{.7\textwidth}
\centering
\includegraphics[trim={30pt 20pt 30pt 30pt},clip, width=\linewidth, keepaspectratio]
{../../repos/lekm/clusters/500-plots-mannen-riktiga-svar-pa-allt-new-distance/K_500_L_1-0/plots/heatmap.png}
\caption{$\gamma=1.0$}
\end{subfigure}%
}\\
\makebox[\linewidth][c]{%
\begin{subfigure}[b]{.7\textwidth}
\centering
\includegraphics[trim={30pt 20pt 30pt 30pt},clip, width=\linewidth, keepaspectratio]
{../../repos/lekm/clusters/500-plots-mannen-riktiga-svar-pa-allt-new-distance/K_500_L_1-6/plots/heatmap.png}
\caption{$\gamma=1.6$}
\end{subfigure}%
\begin{subfigure}[b]{.7\textwidth}
\centering
\includegraphics[trim={30pt 20pt 30pt 30pt},clip, width=\linewidth, keepaspectratio]
{../../repos/lekm/clusters/500-plots-mannen-riktiga-svar-pa-allt-new-distance/K_500_L_2-2/plots/heatmap.png}
\caption{$\gamma=2.2$}
\end{subfigure}%
}
\caption{Feature-weight meshgrid of LEKM (k=500). A row represents a cluster subspace (feature-weight vector), and a column represents a feature-weight along clusters.}
\label{fig:lekm-meshgrid}
\end{figure}
\newpage

\newpage
\begin{figure}[H]
\makebox[\linewidth][c]{%
\begin{subfigure}[b]{.7\textwidth}
\centering
\includegraphics[width=.99\textwidth]{../../repos/lekm/clusters/500-plots-mannen-riktiga-svar-pa-allt-new-distance/K_500_L_0-5/plots/weights-v2.png}
\caption{$\gamma=0.5$}
\end{subfigure}%
\begin{subfigure}[b]{.7\textwidth}
\centering
\includegraphics[width=.99\textwidth]{../../repos/lekm/clusters/500-plots-mannen-riktiga-svar-pa-allt-new-distance/K_500_L_1-0/plots/weights-v2.png}
\caption{$\gamma=1.0$}
\end{subfigure}%
}\\
\makebox[\linewidth][c]{%
\begin{subfigure}[b]{.7\textwidth}
\centering
\includegraphics[width=.99\textwidth]{../../repos/lekm/clusters/500-plots-mannen-riktiga-svar-pa-allt-new-distance/K_500_L_1-6/plots/weights-v2.png}
\caption{$\gamma=1.6$}
\end{subfigure}%
\begin{subfigure}[b]{.7\textwidth}
\centering
\includegraphics[width=.99\textwidth]{../../repos/lekm/clusters/500-plots-mannen-riktiga-svar-pa-allt-new-distance/K_500_L_2-2/plots/weights-v2.png}
\caption{$\gamma=2.2$}
\end{subfigure}%
}
\caption{Distribution of feature weights values upon all clusters ($k=500$)}
\label{fig:lekm-distribution}
\end{figure}
\newpage

\subsection{FSC}
\Cref{fig:fsc-purity} shows how the algorithm performs in terms of purity on the dataset. From the figure we can see that the purity scores keep increasing and that the best score is found where $\lim_{\beta \to{\infty}}$, for our test a limit was set to $\beta=30$ and was therefore the best performing value. The most significant increase occurs between $\beta=1.8$ and $\beta=2.02$ with an increase from $0.29$ to $0.40$.

The amount of iterations, as shown in \cref{fig:fsc-iterations}, are high for smaller values of $\beta$ and equal two for larger $\beta$.

\Cref{fig:fsc-meshgrid} shows how feature weights are distributed along clusters. A selection of a lower $\beta$ ($\beta=1.5$) results in a single dominant feature weight for most clusters, similar to EWKM. As $\gamma$ increased, more and more features were given importance in clusters. Again, note the scale of the meshgrids, some small amount of clusters had larger differences between feature weight values than other clusters, resulting in a more "blue" than "yellow" picture. A choice of $\beta=1.5$ is shown to have feature weights that led to a single dominant weight per feature. Higher values resulted in uniform more dominant features. Similar to LEKM, certain features appeared dominant feature in many clusters, while other features were deemed unimportant for multiple clusters.

In terms of distribution as shown in \cref{fig:fsc-distribution}, the resulting distribution are positively skewed, but like LEKM the distribution ended up being more sharp for larger values of $\beta$.

\subsection{k-means}
k-means converged after seven iterations given $k=500$. The purity score was $45.4\%$ as shown in \cref{table:purity}.

\begin{frame}

\begin{figure}
\begin{minipage}{.45\textwidth}
  \includegraphics[width=\linewidth, keepaspectratio]{../../repos/fsc/clusters/500-plots-mannen-1/plots/gamma-purities.png}
  \caption{Purity accuracy of FSC given various values of $\gamma$ ($k=500$).}
  \label{fig:fsc-purity}
\end{minipage}\hfill
\begin{minipage}{.45\textwidth}
  \includegraphics[width=\linewidth, keepaspectratio]{../../repos/fsc/clusters/500-plots-mannen-1/plots/gamma-iterations.png}
  \caption{Iterations until convergence of FSC given various values of $\gamma$ ($k=500$).}
  \label{fig:fsc-iterations}
\end{minipage}
\end{figure}
\end{frame}


\begin{figure}[H]
  \makebox[\linewidth][c]{
  \begin{subfigure}[b]{.7\textwidth}
  \centering
  \includegraphics[trim={30pt 20pt 30pt 38pt},clip, width=\linewidth, keepaspectratio]{../../repos/fsc/clusters/500-plots-mannen-1/K_500_L_1-5/plots/heatmap.png}
  \caption{$\gamma=1.5$}
  \end{subfigure}
  \begin{subfigure}[b]{.7\textwidth}
  \centering
  \includegraphics[trim={30pt 20pt 30pt 38pt},clip, width=\linewidth, keepaspectratio]
  {../../repos/fsc/clusters/500-plots-mannen-1/K_500_L_2-1/plots/heatmap.png}
  \caption{$\gamma=2.1$}
  \end{subfigure}
  }\\
  \makebox[\linewidth][c]{
  \begin{subfigure}[b]{.7\textwidth}
  \centering
  \includegraphics[trim={30pt 20pt 30pt 38pt},clip, width=\linewidth, keepaspectratio]
  {../../repos/fsc/clusters/500-plots-mannen-1/K_500_L_3-0/plots/heatmap.png}
  \caption{$\gamma=3.0$}
  \end{subfigure}
  \begin{subfigure}[b]{.7\textwidth}
  \centering
  \includegraphics[trim={30pt 20pt 30pt 38pt},clip, width=\linewidth, keepaspectratio]
  {../../repos/fsc/clusters/500-plots-mannen-1/K_500_L_5-0/plots/heatmap.png}
  \caption{$\gamma=5.0$}
  \end{subfigure}
  }
  \caption{Feature-weight meshgrid of FSC ($k=500$). A row represents a cluster subspace (feature-weight vector), and a column represents a feature-weight along clusters.}
  \label{fig:fsc-meshgrid}
\end{figure}

\begin{figure}[H]
  \makebox[\linewidth][c]{
  \begin{subfigure}[b]{.7\textwidth}
  \centering
  \includegraphics[width=.99\textwidth]{../../repos/fsc/clusters/500-plots-mannen-1/K_500_L_1-5/plots/weights-v2.png}
  \caption{$\gamma=1.5$}
  \end{subfigure}
  \begin{subfigure}[b]{.72\textwidth}
  \centering
  \includegraphics[width=.99\textwidth]{../../repos/fsc/clusters/500-plots-mannen-1/K_500_L_2-1/plots/weights-v2.png}
  \caption{$\gamma=2.1$}
  \end{subfigure}
  }\\
  \makebox[\linewidth][c]{
  \begin{subfigure}[b]{.7\textwidth}
  \centering
  \includegraphics[width=.99\textwidth]{../../repos/fsc/clusters/500-plots-mannen-1/K_500_L_3-0/plots/weights-v2.png}
  \caption{$\gamma=3.0$}
  \end{subfigure}
  \begin{subfigure}[b]{.7\textwidth}
  \centering
  \includegraphics[width=.99\textwidth]{../../repos/fsc/clusters/500-plots-mannen-1/K_500_L_5-0/plots/weights-v2.png}
  \caption{$\gamma=5.0$}
  \end{subfigure}
  }
  \caption{Distribution of feature weights values upon all clusters ($k=500$)}
  \label{fig:fsc-distribution}
\end{figure}

\subsection{Summary}
\label{res:summary}
The best parameters, together with their purity scores are shown for $k=50,100,500$ in \cref{table:purity} for each algorithm. The table shows that $k=500$ achieves the best purity scores for all algorithms except EWKM (where $k=100$ results in the greatest purity). For $k=50$ and $k=100$, k-means is the best performing algorithm, but only marginally. For $k=500$ LEKM performs marginally better than k-means with the best overall score of (45.5\%). For all values of $k$, EWKM is the worst performing algorithm.

\begin{table}[h!]
\begin{center}
  \begin{tabular}{lr@{\hspace{0.2in}}rrr@{\hspace{0.2in}}rrrrrrrr}
  \hline\noalign{\smallskip}
  && \multicolumn{2}{c}{\tbtitle{k-means}} && \multicolumn{2}{c}{\tbtitle{EWKM}} && \multicolumn{2}{c}{\tbtitle{LEKM}} && \multicolumn{2}{c}{\tbtitle{FSC}} \\
  \cline{3-4}\cline{6-7}\cline{9-10}\cline{12-13}
        \noalign{\smallskip} \multicolumn{1}{c}{\textit{k}} && & \tbtitle{Purity} && \tbtitle{$\gamma$} & \tbtitle{Purity} && \tbtitle{$\gamma$} & \tbtitle{Purity} && \tbtitle{$\beta$} & \tbtitle{Purity}
\\
  \noalign{\smallskip}
  \hline\noalign{\smallskip}

        \multicolumn{1}{r|}{50 } && & 39.0\% && 0.005 & 28.2\% && 0.0 & 38.8\% && 30.0 & 38.0\\
        \multicolumn{1}{r|}{100} && & 40.8\% && 0.005 & 28.9\% && 2.1 & 40.6\% && 15 & 39.6\\
        \multicolumn{1}{r|}{500} && & 45.4\% && 0.001 & 28.6\% && 1.4 & \textbf{45.5\%} && 30.0 & 43.8\%\\
  \noalign{\smallskip}
  \hline
\end{tabular}
\end{center}
\caption{Best parameter- and purity-score ($\gamma$, $\beta$) given \textit{k}, where scores represent the mean purity.}
\label{table:purity}
\end{table}


\section{External Evaluation}
\label{section:external}
\begin{color}{modified}
The results of the blind test can be found in \cref{table:rating}. The mean scores and standard deviation of existing playlists, k-means and LEKM is shown for each criteria in the first-part of the table. A single-factor ANOVA test is then shown beneath the mean scores.
\end{color}

\begin{color}{modified}
  The clusters sourced from existing playlists were shown to have a higher mean than k-means and LEKM on all measured criteria except \textit{playlist uniqueness}. k-means was shown to have marginally higher mean scores compared to LEKM for all measured criteria. A null hypothesis stating that all sources had the same mean, could however not be rejected for any criterion based on the single-factor ANOVA test shown in the same table. It can therefore not be proven to be any significant difference between the different sources.


% The clusters sourced from existing playlists were shown to have a higher mean than k-means and LEKM on all parameters except novelty. However, as shown in \cref{table:rating} an ANOVA test could not reject a null hypothesis of all sources having the same mean. For \textit{General Quality}, there was a score difference between playlists (7.7) and the algorithms (k-means - 6.6, and LEKM - 6.4), still the ANOVA test could not show enough difference to reject a null hypothesis. Both algorithms were rated better in comparison to playlists, on \textit{Audio Similarity} than \textit{Cultural Similarity}. Regarding k-means vs. LEKM, k-means mean values were marginally higher than LEKM in all categories, but cannot be concluded to be significantly better based on the ANOVA test.
\end{color}

There were also results obtained from the discussions of the blind test. Clusters from the algorithms were mentioned to have a core of songs that had an \textit{obvious} theme. In addition to the core, there were two or three songs in the shown sample of the songs, that were \textit{culturally} different according to the composers, i.e. the songs' artists had a different audience than the core songs. For well performing clusters --- \begin{color}{modified} in terms of \textit{General Quality} scores \end{color} --- that difference added depth to the playlist in a positive fashion, e.g. adding reggae to a hip-hop cluster . For bad performing clusters the mixins were off-putting for the theme of the cluster and made the cluster worse in terms of general quality, but also in terms of \textit{Playlist Uniqueness}, e.g. adding country songs to an indie rock core.

\begin{table}[H]
\makebox[\linewidth][c]{
\begin{tabular}{l@{\hspace{0.2in}}rrrrr}
  \hline\noalign{\smallskip}
  % \multicolumn{1}{l}{} & \multicolumn{4}{c}{\tbtitle{Evaluation}}\\
  \multicolumn{1}{l}{} & \multicolumn{4}{c}{\tbtitle{$\bar{x} (\sigma)$ }}\\
  \noalign{\smallskip}\cline{2-5}\noalign{\smallskip}
  \multicolumn{1}{l}{\tbtitle{Source}} & \tbtitle{General Quality} & \tbtitle{Audio Similarity} & \tbtitle{Cultural Similarity} & \tbtitle{Playlist Uniqueness}
  \\
\noalign{\smallskip}
  \hline
  \noalign{\smallskip}
  \textit{Existing Playlist} & 7.7 (1.2)	&   7.8	(1.5)    &   7.7 (0.6)	&   4.5	(3.5) \\
  \noalign{\smallskip}
  \hline
  \noalign{\smallskip}
  \textit{k-means} & 6.6 (2.5)	&   7.0	(2.3)    &   6.2 (2.9)	&   5.4	(3.2) \\
  \textit{LEKM} & 6.4 (3.4)	&   6.4	(2.4)    &   6.0 (2.5)	&   3.8	(2.6) \\
  \noalign{\smallskip}
  \hline
  \noalign{\smallskip}
  \noalign{\smallskip}
  \multicolumn{1}{l}{} & \multicolumn{4}{c}{\tbtitle{\emph{\color{modified}{ANOVA}}}} & \\
  \noalign{\smallskip}\cline{2-5}\noalign{\smallskip}
  \multicolumn{1}{l}{\tbtitle{Index}} & \tbtitle{General Quality} & \tbtitle{Audio Similarity} & \tbtitle{Cultural Similarity} & \tbtitle{Playlist Uniqueness}
  \\
  \noalign{\smallskip}
  \hline
  \noalign{\smallskip}
  $F_{0.05}$    & 0.21 &    0.45    &   0.47	&   0.37 \\
  $F_{crit}$    & 4.10 &    3.98    &   4.10	&   3.98 \\
  $P_{value}$   & 0.81 &    0.65    &   0.64	&   0.70 \\
  \hline
\end{tabular}
}
\caption{Mean rating and standard deviation (in brackets) of five clusters given source type, along with single factor ANOVA scores that show that we cannot reject $H_{0}$ for any criterion.}
\label{table:rating}
\end{table}


% \begin{figure}[h]
% \begin{tikzpicture}
%     \begin{axis}[
%         width  = 0.85*\textwidth,
%         height = 8cm,
%         major x tick style = transparent,
%         ybar=2*\pgflinewidth,
%         bar width=14pt,
%         ymajorgrids = true,
%         ylabel = {Rating (1-10)},
%         symbolic x coords={Playlist,k-means,LEKM},
%         xtick = data,
%         scaled y ticks = false,
%         enlarge x limits=0.25,
%         ymin=0,
%         legend style={
%           at={(1.25,0.5)},
% 	  anchor=south,
%           legend columns=1
%         }
%     ]
%       \addplot
%         coordinates {(Playlist, 7.0) (k-means,6.6) (LEKM,6.4)};
%       \addplot
%         coordinates {(Playlist, 7.6) (k-means,7.0) (LEKM,6.4)};
%       \addplot
%         coordinates {(Playlist, 8.0) (k-means,6.2) (LEKM,6.0)};
%       \addplot
%         coordinates {(Playlist, 5.6) (k-means,5.4) (LEKM,3.8)};
%       \legend{General Quality, Audio Similarity, Cultural Similarity, Playlist Uniqueness}
%     \end{axis}
% \end{tikzpicture}
% \caption{Average rating of clusters given source type}
% \label{fig:rating}
% \end{figure}





\end{document}

