\documentclass[../report.tex]{subfiles}

\begin{document}
\chapter{Conclusions}
\begin{color}{modified}
  Multiple SSC algorithms were tested on the given high-dimensional musical dataset --- sampled to a size of $50k$, and processed to only hold numerical features. LEKM turned out to be the best performing \textit{SSC}-algorithm in terms of genre purity. Comparisons between k-means (a traditional algorithm) and LEKM show that there is no significant performance difference between the algorithms, i.e. feature-weighting did not improve the performance, possibly due to the nature of the dataset. Similar \textit{Purity} scores were found for both algorithms. Based on a panel of judges, any of the two algorithms could be used to produce clusters of general quality similar to existing playlists. Based on statistical analysis, similar performance were found for k-means and SSC. The results indicate that the generated clusters could be used as a tool when composing new playlists on the musical library that is our dataset. Future work should see a comparison between k-means and LEKM on a larger sample of the dataset, to see if there is a benefit for feature weighting on a larger scale. A larger sample would need the code of LEKM to be rewritten for a distributed network. From a broader perspective on how SSC-algorithms performs on high-dimensional datasets, another dataset comparison is needed.
\end{color}
\end{document}

