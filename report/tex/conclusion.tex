\documentclass[../report.tex]{subfiles}

\begin{document}
\chapter{Conclusions}
The given high-dimensional musical dataset --- sampled to a size of $50k$, and processed to only hold numerical features --- had LEKM as the best performing \textit{SSC}-algorithm. The comparisons done between \textit{K-means} and LEKM show that there is no significant performance difference between the algorithms, i.e. feature-weighting did not improve the performance. Similar \textit{Purity} scores were found for both algorithms. Based on a panel of judges, any of the two algorithms could be used to produce clusters of high general quality and audio similarity, although K-means performed better in terms of novelty. The results indicate that the generated clusters could be used as a tool when composing new playlists based on the dataset. Future work should see a comparison between K-means and LEKM on a larger sample of the dataset, to see if there is a benefit for feature weighting on a larger scale. A larger sample would need the code of LEKM to be rewritten for a distributed network. From a broader perspective on how SSC-algorithms performs on high-dimensional datasets, another dataset comparison is needed.
\end{document}

