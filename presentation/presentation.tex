\documentclass{beamer}

\definecolor{KTHblue}{RGB}{25 84 166}
\definecolor{KTHlightblue}{RGB}{46 171 224}
\definecolor{KTHgreen}{RGB}{163 207 68}
\definecolor{KTHpink}{RGB}{227 89 171}
\definecolor{KTHgrey}{RGB}{103 108 118}

\usetheme[titleformat=smallcaps, sectionpage=contents, progressbar=frametitle, block=fill]{metropolis}
\title{Soft-Subspace Clustering on a High-Dimensional Musical Dataset}
\setbeamercolor{normal text}{fg=KTHblue, bg=white}
\setbeamercolor{alerted text}{fg=KTHgreen, bg=purple}
\setbeamercolor{example text}{fg=KTHpink, bg=KTHgrey}
\setbeamercolor{progress bar}{fg=KTHpink}
\setbeamerfont{examiner}{size=\fontsize{8pt}{8pt}}
\setbeamerfont{supervisor}{size=\fontsize{8pt}{8pt}}
\setsansfont[BoldFont={Fira Sans SemiBold}]{Fira Sans Book}
% Use metropolis theme

% -------------------------------------

\date{\today}
\author{Emil Juzovitski}
\examiner{Pawel Herman}
\supervisor{Johan Gustavsson}
\institute{Master Thesis Presentation}
\begin{document}
\maketitle

% -------------------------------------

\begin{frame}{Table of Contents}
  \tableofcontents
\end{frame}

% -------------------------------------

\section{Introduction}

% -------------------------------------

\begin{frame}{Clustering Analysis}
\begin{block}{What is clustering analysis about?}
Finding \alert{clusters} (groups) in a set of data objects \pause , with \alert{feature similar} data objects partitioned into the same cluster \pause , and \alert{feature dissimilar} data objects partitioned into different cluster.
\end{block}
\end{frame}

% -------------------------------------

\begin{frame}{Clustering Analysis}
\begin{columns}
  \column{.5\textwidth}
    \begin{block}{Clustering analysis}
    Finding \alert{clusters} (groups) in a set of data objects.
    \end{block}
  \column{.5\textwidth}
    \begin{itemize}
      \item<2> \alert{feature similar} data objects partitioned into the same cluster
      \item<2> \alert{feature dissimilar} data objects partitioned into different cluster.
    \end{itemize}
\end{columns}
\end{frame}

% -------------------------------------

\begin{frame}{Clustering Analysis}
    \begin{block}{Clustering analysis}
    Finding \alert{clusters} (groups) in a set of data objects \only<3->{, based on \alert{similarity}}
    \end{block}
    \only<2>{
    \begin{itemize}
      \item \alert{feature similar} data objects partitioned into the same cluster
      \item \alert{feature dissimilar} data objects partitioned into different cluster.
    \end{itemize}
    }
    \only<3>{Example Tasks:
    \begin{itemize}
      \item Fitting products into different aisles in a grocery store
      \item Grouping distributors based on the products they sell
    \end{itemize}
    }
    \only<4->{
      How do we measure similarity?
      % With a distance measure.
    }

    \only<5->{
      By defining a \alert{distance} measurement between points: 
      % With a distance measure.
    }

    \begin{center}
      \only<6->{\alert{$D(X_1,X_2)$}}
      \only<7>{
        $= \sum_{j}^{m}{d(x_{1j},x_{2j})}$
      }
    \end{center}

\end{frame}

% -------------------------------------


\begin{frame}[fragile]
\frametitle{An Algorithm For Finding Primes Numbers.}
\begin{semiverbatim}
\uncover<1->{\alert<0>{int main (void)}}
\uncover<1->{\alert<0>{\{}}
\uncover<1->{\alert<1>{ \alert<4>{std::}vector<bool> is_prime (100, true);}}
\uncover<1->{\alert<1>{ for (int i = 2; i < 100; i++)}}
\uncover<2->{\alert<2>{
if (is_prime[i])}}
\uncover<2->{\alert<0>{
\{}}
\uncover<3->{\alert<3>{
\alert<4>{std::}cout << i << " ";}}
\uncover<3->{\alert<3>{
for (int j = i; j < 100;}}
\uncover<3->{\alert<3>{
is_prime [j] = false, j+=i);}}
\uncover<2->{\alert<0>{
\}}}
\uncover<1->{\alert<0>{ return 0;}}
\uncover<1->{\alert<0>{\}}}
\end{semiverbatim}
\visible<4->{Note the use of \alert{\texttt{std::}}.}
\end{frame}
% -------------------------------------

\begin{frame}{Features}
\begin{itemize}
  \item Describes a property of a object/point \pause
  \item An object e.g. A shoe, can be represented by features such as \alert{brand} (nike, adidas), \alert{style} (sneaker, flip-flops, leather), and \alert{cost} (\$) \pause
  \item \alert{brand} is \alert{categorical}
  \item \alert{cost} is \alert{numerical}

\end{itemize}
\end{frame}

% -------------------------------------

\section{Background}

% -------------------------------------

\begin{frame}{Third Frame}
Hello, world!
\begin{example}
  Hello
\end{example}
\end{frame}

% -------------------------------------

\begin{frame}[standout]
  Thank you!
\end{frame}

% -------------------------------------

\section{Method}

% -------------------------------------

\begin{frame}{fourth Frame}
Hello, world!
\end{frame}

% -------------------------------------

\section{Results}

% -------------------------------------

\section{Discussion}

% -------------------------------------

\begin{frame}{fourth Frame}
Hello, world!
\end{frame}

\end{document}
